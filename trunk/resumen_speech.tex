% Tamaño de letra.
\documentclass[12pt,titlepage]{report}

%------------------------------ Paquetes ----------------------------------

% Paquetes:

%Para comentarios multilínea.
\usepackage{verbatim}

% Para tener cabecera y pie de página con un estilo personalizado.
\usepackage{fancyhdr}

% Codificación UTF-8
\usepackage[utf8]{inputenc}

% Castellano.
\usepackage[spanish]{babel}

% Tamaño de página y márgenes.
\usepackage[a4paper,headheight=16pt,scale={0.75,0.8},hoffset=0.5cm]{geometry}

% Para poder agregar notas al pie en tablas:
%\usepackage{threeparttable}

% Tipo de letra Helvetica (Arial).
%\usepackage{helvet}
%\renewcommand\familydefault{\sfdefault}

% Gráficos:

% Para incluir imágenes, el siguiente código carga el paquete graphicx
% según se esté generando un archivo dvi o un pdf (con pdflatex).

% Para generar dv.
%\usepackage[dvips]{graphicx}

% Para generar pdf.
\usepackage[pdftex]{graphicx}
\pdfcompresslevel=9

\usepackage{pdfpages}

%
% Directorio donde están las imagenes.
%
%\newcommand{\imgdir}{includes}
%\graphicspath{{\imgdir/}}

%------------------------------ ~paquetes ---------------------------------

%------------------------- Inicio del documento ---------------------------
\begin{document}
% -------------------------- Título y autor(es) ---------------------------

\title{Información en las organizaciones}
\author{}

\part{Resumen para Speech}
\section{Historia}

En sus inicios (IEP) Industrias Electrot\'ecnicas Puig  se especializaba en la fabricaci\'on de aparatos reflectores para alumbrado en la ciudad española de Barcelona.

En el año 1966 pasa a formar parte del Grupo Simon, un holding de empresas del mercado el\'ectrico español con visionaria expansi\'on hacia a los cinco continentes.

A principios de la d\'ecada del 90 adapta nuevamente su estructura y su imagen, empezando a conoc\'ersela como IEP DE ILUMINACI\'ON.

Finalmente, es en el año 1998 que IEP DE ILUMINACI\'ON llega a la Argentina, convirti\'endose en el lapso de 12 años en la principal responsable de la comercializaci\'on de luminarias para toda Am\'erica del Sur.

El Grupo Simon fue incorporando nuevos centros de producci\'on y actualmente su presencia mundial alcanza a m\'as de 50 pa\'ises, con sede central en Barcelona (España).


\section{Ubicaci\'on y caracter\'isticas de la planta}
La empresa se encuentra ubicada en el conurbano bonaerense, en el kil\'ometro 37 del ramal Escobar de la Ruta Panamericana

Durante sus primeros años de actividad en el pa\'is, la empresa cont\'o con planta de producci\'on en la localidad de Munro (Buenos Aires) y oficinas comerciales en San Isidro (Buenos Aires), pero el creciente aumento de los vol\'umenes de venta fundados en la producci\'on de luminarias de avanzada tecnolog\'ia y diseños, demand\'o la instalaci\'on de una planta fabril de mayor tamaño y capacidad productiva.

IEP DE ILUMINACI\'ON cuenta desde el año 2005 con instalaciones propias dentro del Centro Industrial Gar\'in (Buenos Aires), garantizando rapidez operativa y de organizaci\'on al reunir en un mismo lugar tanto sus \'areas Comerciales como las de Producci\'on y Almacenamiento.

La empresa cuenta con su propio Laboratorio de Luminotecnia para asegurar la m\'axima calidad y seguridad en sus productos.

Brinda, adem\'as, Capacitaci\'on y seminarios a profesionales e interezados del tema.

\section{Productos que ofrece}
En grandes razgos se trata de una empresa dedicada a la fabricaci\'on y comercializaci\'on de luminarias, farolas, columnas y soportes; ofreciendo soluciones de buena calidad en Alumbrado P\'ublico, \'Areas Verdes, Alumbrado Industrial, Alumbrado Interior, e incluso en Iluminaci\'on con Sumergibles y Leds.

\end{document}
